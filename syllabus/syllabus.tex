% Options for packages loaded elsewhere
\PassOptionsToPackage{unicode}{hyperref}
\PassOptionsToPackage{hyphens}{url}
\PassOptionsToPackage{dvipsnames,svgnames,x11names}{xcolor}
\documentclass[
  twoside]{article}
\usepackage{xcolor}
\usepackage[left=1.0in, right=1.0in, top=0.8in, bottom=0.4in]{geometry}
\usepackage{amsmath,amssymb}
\setcounter{secnumdepth}{-\maxdimen} % remove section numbering
\usepackage{iftex}
\ifPDFTeX
  \usepackage[T1]{fontenc}
  \usepackage[utf8]{inputenc}
  \usepackage{textcomp} % provide euro and other symbols
\else % if luatex or xetex
  \usepackage{unicode-math} % this also loads fontspec
  \defaultfontfeatures{Scale=MatchLowercase}
  \defaultfontfeatures[\rmfamily]{Ligatures=TeX,Scale=1}
\fi
\usepackage{lmodern}
\ifPDFTeX\else
  % xetex/luatex font selection
\fi
% Use upquote if available, for straight quotes in verbatim environments
\IfFileExists{upquote.sty}{\usepackage{upquote}}{}
\IfFileExists{microtype.sty}{% use microtype if available
  \usepackage[]{microtype}
  \UseMicrotypeSet[protrusion]{basicmath} % disable protrusion for tt fonts
}{}
\makeatletter
\@ifundefined{KOMAClassName}{% if non-KOMA class
  \IfFileExists{parskip.sty}{%
    \usepackage{parskip}
  }{% else
    \setlength{\parindent}{0pt}
    \setlength{\parskip}{6pt plus 2pt minus 1pt}}
}{% if KOMA class
  \KOMAoptions{parskip=half}}
\makeatother
\usepackage{graphicx}
\makeatletter
\newsavebox\pandoc@box
\newcommand*\pandocbounded[1]{% scales image to fit in text height/width
  \sbox\pandoc@box{#1}%
  \Gscale@div\@tempa{\textheight}{\dimexpr\ht\pandoc@box+\dp\pandoc@box\relax}%
  \Gscale@div\@tempb{\linewidth}{\wd\pandoc@box}%
  \ifdim\@tempb\p@<\@tempa\p@\let\@tempa\@tempb\fi% select the smaller of both
  \ifdim\@tempa\p@<\p@\scalebox{\@tempa}{\usebox\pandoc@box}%
  \else\usebox{\pandoc@box}%
  \fi%
}
% Set default figure placement to htbp
\def\fps@figure{htbp}
\makeatother
\setlength{\emergencystretch}{3em} % prevent overfull lines
\providecommand{\tightlist}{%
  \setlength{\itemsep}{0pt}\setlength{\parskip}{0pt}}
\usepackage[charter]{mathdesign}
\usepackage{fancyhdr}
\usepackage{titlesec}
\usepackage{lastpage}
\usepackage{hyperref}
\hypersetup{colorlinks=false}
\pagestyle{fancy}
\lhead{MA-110 Syllabus, Hansen \the\year{}}
\rhead{Page \thepage{} of \pageref*{LastPage}}
\cfoot{}
\setlength{\headsep}{0.2in}

\titleformat{\section}{\LARGE\bfseries}{\thesection}{1em}{}
\titleformat{\subsection}{\Large\bfseries}{\thesection}{1em}{}
\titleformat{\subsubsection}
   {\large\bfseries}{\thesection}{1em}{}[\hrule width\linewidth]

\usepackage{enumitem}

\setenumerate{leftmargin=*}

\newcommand{\vect}[1]{\mathbf{#1}} % vectors in bold face

%\newcommand{\vect}[1]{\mathbf{#1}\,} % vectors in bold face (need thin
                                     % space after)
%\newcommand{\vect}[1]{\vec{#1}} %vectors as arrows

\providecommand{\norm}[1]{\left\lvert#1\right\rvert} % norms as single lines
%\providecommand{\norm}[1]{\left\lVert#1\right\rVert} % norms as double lines

% bold or blackboard bold!?
\newcommand{\NN}{\mathbb{N}}
%\newcommand{\RR}{\mathbf{R}}
\newcommand{\RR}{\mathbb{R}}

% some useful abbreviations
\newcommand{\vx}{\vect{x}}
\newcommand{\vy}{\vect{y}}
\newcommand{\vz}{\vect{z}}
\newcommand{\vu}{\vect{u}}
\newcommand{\vv}{\vect{v}}
\newcommand{\vw}{\vect{w}}
\newcommand{\va}{\vect{a}}
\newcommand{\vb}{\vect{b}}
\newcommand{\vc}{\vect{c}}
\newcommand{\ve}{\vect{e}}
\newcommand{\vf}{\vect{f}}
\newcommand{\vF}{\vect{F}}
\newcommand{\vg}{\vect{g}}
\newcommand{\vh}{\vect{h}}
\newcommand{\vl}{\vect{l}}
\newcommand{\vm}{\vect{m}}
\newcommand{\vn}{\vect{n}}
\newcommand{\vp}{\vect{p}}
\newcommand{\vr}{\vect{r}}
\newcommand{\vs}{\vect{s}}
\newcommand{\vi}{\vect{i}}
\newcommand{\vj}{\vect{j}}
\newcommand{\vk}{\vect{k}}
\newcommand{\vzero}{\vect{0}}
% lower-case greek letters are handled differently: poor man's bold macro
%\newcommand{\vphi}{\pmb{\phi}}

\newcommand{\malename}{Westley} % recurring male name
\newcommand{\femalename}{Buttercup} % recurring female name
\usepackage{booktabs}
\usepackage{longtable}
\usepackage{array}
\usepackage{multirow}
\usepackage{wrapfig}
\usepackage{float}
\usepackage{colortbl}
\usepackage{pdflscape}
\usepackage{tabu}
\usepackage{threeparttable}
\usepackage{threeparttablex}
\usepackage[normalem]{ulem}
\usepackage{makecell}
\usepackage{xcolor}
\usepackage{bookmark}
\IfFileExists{xurl.sty}{\usepackage{xurl}}{} % add URL line breaks if available
\urlstyle{same}
\hypersetup{
  colorlinks=true,
  linkcolor={blue},
  filecolor={Maroon},
  citecolor={Blue},
  urlcolor={blue},
  pdfcreator={LaTeX via pandoc}}

\author{}
\date{\vspace{-2.5em}}

\begin{document}

\section[Abstract Algebra (MA-110) Westmont College, Fall
2025]{\texorpdfstring{Abstract Algebra (MA-110) Westmont College, Fall
2025\footnote{Large parts of this syllabus are borrowed from previous
  syllabi by David Hunter.}}{Abstract Algebra (MA-110) Westmont College, Fall 2025}}\label{abstract-algebra-ma-110-westmont-college-fall-20252}

\begin{tabular}[t]{ll}
\toprule
Time & MWF, 8:00am-9:05am\\
Location & Winter Hall 110\\
Professor & Kyle Hansen, Ph.D.\\
Email & kylhansen@westmont.edu\\
Office & Winter Hall 303\\
\addlinespace
Office Hours & TBA\\
\bottomrule
\end{tabular}

\subsection{Course Description}\label{course-description}

\subsubsection{(Four credit hours), Prerequisite(s): MA-020 Linear
Algebra.}\label{four-credit-hours-prerequisites-ma-020-linear-algebra.}

Abstract algebraic patterns pervade modern mathematics. Given simple
definitions of groups, rings, and fields, this course develops the
intricate theory of these objects, including permutation groups,
subgroups, quotient groups and rings, isomorphisms, and extension
fields.

\subsubsection{Put another way\ldots{}}\label{put-another-way}

This class is the study of the operations of multiplication and addition
(and, secretly, function composition) set in their most abstract form.
Starting with only and handful of definitions and even fewer
assumptions, we proceed axiomatically to deduce a large volume of
powerful mathematical results. The material in this course (and the
sequel, MA-111) represents the consequences of some very low-level
mathematical foundations: logic, sets, and the integers. These ideas
have far-reaching applications throughout pure and applied mathematics,
and are part of the \emph{lingua franca} of the discipline. Throughout
this semester and the next (the highlight of this course), you will be
guided through a series of deductive discoveries along the path that
modern mathematicians have blazed. Whether this path is the only
reasonable one will be yours to judge once you have completed the
journey.

``\emph{But where's the quadratic formula?}'' you ask. Indeed,
historically, much of this subject was motivated by the search for
solutions to polynomial equations: are there analogs to the quadratic
formula for polynomials of degree greater than 2? We take our time to
get there, but eventually our trajectory will lead us into study of
roots of polynomial equations, as you might expect, but keep your eyes
peeled and your head up---your perspective on these innocent-looking
functions will be pushed into deeper, higher, more abstract realms of
the subject.

\subsection{Learning Outcomes}\label{learning-outcomes}

\subsubsection{Institutional Learning Outcomes
(ILO's)}\label{institutional-learning-outcomes-ilos}

The faculty of Westmont College have established common learning
outcomes for all courses at the institution. These outcomes are
summarized as follows: 1. Christian Understanding, Practices, and
Affections, 2. Global Awareness, 3. Diversity, 4. Critical Thinking, 5.
Quantitative Literacy, 6. Competence in Written Communication, 7.
Competence in Oral Communication, and 8. Information Literacy.

\subsubsection{Program Learning Outcomes
(PLO's)}\label{program-learning-outcomes-plos}

Additionally, the mathematics department at Westmont College has
formulated the following learning outcomes for all of its classes.

\begin{enumerate}
\def\labelenumi{\arabic{enumi}.}
\tightlist
\item
  \textbf{Core Knowledge:} Students will demonstrate knowledge of the
  main concepts, skills, and facts of the discipline of mathematics.
\item
  \textbf{Communication:} Students will be able to communicate
  mathematical ideas following the standard conventions of writing or
  speaking in the discipline.
\item
  \textbf{Creativity:} Students will demonstrate the ability to
  formulate and make progress toward solving non-routine problems.
\item
  \textbf{Christian Connection:} Students will incorporate their
  mathematical skills and knowledge into their thinking about their
  vocations as followers of Christ.
\end{enumerate}

\subsubsection{Course Learning Outcomes
(CLOs)}\label{course-learning-outcomes-clos}

The above outcomes are reflected in the particular learning outcomes for
this course. After taking this course, you should be able to

\begin{enumerate}
\def\labelenumi{\arabic{enumi}.}
\tightlist
\item
  Demonstrate understanding of modern abstract algebra. (PLO 1, ILOs
  3,4)
\item
  Write mathematical arguments according to the standards of the
  discipline. (PLO 2, ILOs 3,5)
\item
  Construct solutions to novel problems, demonstrating perseverance in
  the face of open-ended or partially-defined contexts. (PLO 3, ILO 3)
\item
  Consider the theological implications of the subject matter. (PLO 4,
  ILO 1)
\end{enumerate}

These outcomes will be assessed by written prework assignments, typeset
final drafts, and written exams, as described
\hyperref[course-logistics]{below}.

\subsubsection{General Education Writing Intensive
Course}\label{general-education-writing-intensive-course}

In addition to the above Learning Outcomes, this course is classified as
``writing intensive'' for the purposes of Westmont's general education
requirements. Therefore we will spend a significant amount of time
learning how to write mathematically. Your written prework assignments
will contain many problems where you are asked to ``prove'' something,
and I will grade these proofs taking both correctness and exposition
into account. A selection of problems will be assigned for you to
typeset carefully and submit a final draft, doing revisions as
necessary. I expect you all to make progress in your ability to write
mathematically.

\subsection{Course Logistics}\label{course-logistics}

\subsubsection{Textbook}\label{textbook}

The text for this course is \emph{Abstract Algebra I \& II}, by David
Hunter, made freely available and used with his permission. These are a
revised and expanded version of \emph{Introductory Abstract Algebra}, by
Margaret L. Morrow, \emph{Journal of Inquiry Based Learning in
Mathematics} \textbf{26}, (2012). Morrow's notes provide a gentle,
inquiry-based introduction to group theory, with Hunter adding several
chapters which cover rings and fields.\footnote{This is not a
  ``traditional'' textbook, and requires your active participation in
  its completion. Throughout the semester you will be adding pages of
  notes, prework, and final drafts to your binder, which will evolve
  into a complete text that you have taken part in writing.}

\subsubsection{Grades}\label{grades}

To assess how well we meet some of the designated Learning Outcomes, you
will receive regular grades on your written prework and typeset final
drafts. You will find grades and feedback on Canvas.\footnote{\emph{Caveat}:
  Canvas doesn't always calculate grades properly.} In addition, there
will be four traditional exams and weekly quizzes. Grades are weighted
as follows.

\begin{tabular}[t]{ll}
\toprule
Written Prework Assignments & 15\%\\
Presentations and Participation & 10\%\\
Typeset Final Drafts & 15\%\\
Quizzes (top 10) & 1\% each (10\% total)\\
Exams (3) & 5\% + 10\% + 15\% (30\% total)\\
\addlinespace
Final Exam & 20\%\\
\bottomrule
\end{tabular}

\subsubsection{Written Work \& Participation
(40\%)}\label{written-work-participation-40}

We will work through a series of problems that, along with the sequel,
give a comprehensive coverage of two semesters of advanced undergraduate
or beginning graduate algebra. Ideally, you will supply the answers.
Your written accounts of your inquiry and our class discussions will
produce a complete modern algebra textbook of your own construction.

This course is meant to prepare you for your future studies and work.
You will inevitably encounter mathematical concepts in situations where
nobody is there to explain things to you. I want you to develop the
necessary skills of inquiry to prepare you for these encounters.
Therefore, I will often be expecting you to find answers to problems
yourself. There will be times when I intentionally leave things
unexplained for you to figure out. This is part of an educational
strategy designed to make you a better thinker.

Unless you are told otherwise, \textbf{do not use outside resources}
(e.g., Internet, books, LLMs) when you work on the assignments. I would
rather see a half-right solution that you have constructed yourself than
a perfect solution that somebody else constructed.

This course presents an opportunity for you to grow in your ability to
work by yourself. I strongly recommend that you
\textbf{spend time alone with the assigned problems} before working with
others. If you do work with others on an assignment, please include a
note on your assignment indicating the names of those you worked with.
And remember that
\textbf{the work you hand in should represent your own understanding.}

There will typically be three phases to the inquiry process, all focused
on the sequence of problems from the text.

\paragraph{Prework (15\%)}\label{prework-15}

On most days, there will be a written assignment due the night before
class. My expectation is that these prework assignments will be hand
written; I encourage you to find an efficient notebook system that
allows you to intersperse hand-outs and your own written work. A
three-ring binder works well for this purpose.

Use a scanner app (e.g., Genius Scan, Adobe Scan) to make a PDF of your
prework and submit it on Gradescope. If your prework spans multiple
pages, please combine all pages into a single PDF file. Be sure that
each page is properly matched and in the correct orientation.

In general, you can receive full credit for less-than-perfect prework
assignments, as long as you make significant progress on each problem.
Making progress can include writing down the first and last lines of the
proof, and formulating some questions about what you didn't understand.

\paragraph{Class Discussion (10\%)}\label{class-discussion-10}

The bulk of our class time will be spent discussing the assigned
problems that you have attempted in the prework phase. On most days, you
will be assigned one or more of the problems to present at the board.
First, students write their solutions (or partial solutions) on the
board. Once everyone has finished, we will discuss all the problems
together in whole-class discussion. During these discussions you should
focus on correcting and completing the proofs from your prework
assignments. Since there is not a traditional textbook for this class,
you have the responsibility of recording your own complete solutions for
every problem.

\paragraph{Final Drafts (15\%)}\label{final-drafts-15}

Each week, a selection of problems will be assigned for you to typeset a
final draft of solutions. These final drafts will be graded on logic,
exposition, and formatting. You should add your graded and corrected
final drafts to your binder, which will grow into a textbook of your own
construction.

\subsubsection{Assessments (60\%)}\label{assessments-60}

Research shows that one cannot understand a subject unless one can
recall the necessary material with ease. One most easily recalls
material when one has been given practice recalling it.

\paragraph{Quizzes (10\%)}\label{quizzes-10}

Weekly quizzes at the end of class on Friday will give you practice
recalling definitions or performing simple calculations covered from the
prework material of the previous week (Friday through Wednesday). At
most, your top 10 quizzes will be graded; all others will be dropped.

\paragraph{Exams (50\%)}\label{exams-50}

This course will have three (3) midterm exams and one (1) final exam.
These exams will not be rescheduled, and absences will count as 0
points. Exams will (naturally) be cumulative, loosely paired with each
unit. To reflect this cumulative nature and to help you develop a
``growth mindset'' throughout this course, exam percentages will
increase as the term progresses.

\subsubsection{Unit Schedule}\label{unit-schedule}

A separate document will be provided which outlines the specific dates
and assignments due throughout the course. We will attempt to adhere to
the following general schedule for this course, with exams punctuating
the end of each unit:

\paragraph{Unit 1: Elementary Group
Theory}\label{unit-1-elementary-group-theory}

\begin{itemize}
\tightlist
\item
  Definitions (§1), Constructions (§2), and Homomorphisms (§3) of
  Groups.
\item
  Exam (5\%) on September 29.
\end{itemize}

\paragraph{Unit 2: Structure in
Groups}\label{unit-2-structure-in-groups}

\begin{itemize}
\tightlist
\item
  Cosets (§4), Quotients (§5.1-§5.2), and Isomorphism Theorems (§5.3) of
  Groups.
\item
  Exam (10\%) on October 27.
\end{itemize}

\paragraph{Unit 3: Rings, Ideals, and
Fields}\label{unit-3-rings-ideals-and-fields}

\begin{itemize}
\tightlist
\item
  Elementary Theory (§6), Quotients (§7), and Divisibility (§8) of
  Rings.
\item
  Exam (15\%) on November 24.
\end{itemize}

\paragraph{Polynomial Rings and Field
Extensions}\label{polynomial-rings-and-field-extensions}

\begin{itemize}
\tightlist
\item
  Polynomials (§9.1), Prime \& Maximal Ideals (§9.2-§9.3), and Field
  Extensions (§9.4)
\item
  Final exam (20\%) on December 18.
\end{itemize}

\subsection{Resources, Policies, and
Expectations}\label{resources-policies-and-expectations}

\subsubsection{Attendance}\label{attendance}

One of the primary resources you have in this class is the class time
itself. If you miss a significant number of classes, you will almost
definitely do poorly in this class. If you miss more than six classes
without a valid excuse, I reserve the right to terminate you from the
course with a failing grade.

\subsubsection{Assignment and
Assessments}\label{assignment-and-assessments}

All assignments and assessments must be turned in by the start of class
on the day they are due. Work missed (including tests) without a valid
excuse will receive a zero.

\subsubsection{A.I.}\label{a.i.}

\paragraph{Academic Integrity}\label{academic-integrity}

God called us to co-create with Him as a part of our role in bearing His
image. Collaboration---in which multiple people labor side-by-side
towards a common goal---is a part of God's good design for community and
relationship. Learning communities function best when students have
academic integrity. Cheating is primarily an offense against your
classmates because it undermines our learning community. Therefore,
dishonesty of any kind may result in loss of credit for the work
involved and the filing of a report with the Provost's Office. Major or
repeated infractions may result in dismissal from the course with a
grade of ``F''. Be familiar with the College's plagiarism policy, found
at
\url{https://www.westmont.edu/office-provost/academic-program/academic-integrity-policy}.

In particular, providing someone with an electronic copy of your work is
a breach of the academic integrity policy. Do not email, post online, or
otherwise disseminate any of the work that you do in this class. If you
keep your work on a repository, make sure it is private. You may work
with others on the assignments, but make sure that you write or type up
your own answers yourself. You are on your honor that the work you hand
in represents your own understanding.

\paragraph{Artificial Intelligence}\label{artificial-intelligence}

This course is meant to build your perseverance (see CLOs 1,3). Any use
of LLMs (etc.) will be robbing yourself of your education, and will
almost certainly naturally cause you to perform poorly in this course.
Any unapproved use of LLMs (etc.) will constitute a breach of academic
integrity.

\subsubsection{Accommodations and
Accessability}\label{accommodations-and-accessability}

Westmont is committed to ensuring equal access to academic courses and
college programs. In keeping with this commitment under the Americans
with Disabilities Act (ADA)of 1990, Section 504 of the Rehabilitation
Act of 1973, and the Americans with Disabilities Amendments Act (ADAAA)
of 2008, individuals with diagnoses that impact major life activities
are protected from discrimination and are entitled to reasonable
accommodations. Students who choose to disclose a disability are
encouraged to contact the Accessibility Resource Office (ARO) as early
as possible in the semester to discuss potential accommodations for this
course. Accommodations are designed to ensure equal access to programs
for all students who have a disability that impacts their participation
in college activities. Email
\href{mailto:aro@westmont.edu}{\nolinkurl{aro@westmont.edu}} or see
\url{https://www.westmont.edu/accessibility-resources} for more
information.

\end{document}
